\newpage
\section{Input Files}
\label{sec:input}
This section contains information on the format of \Langmuir input files.
Input files are just text files that you can edit with any text editor.
However, be warned that input files can be very long if they contain
    information on traps.
In this case, it is best to use a text editor capable of dealing with
    very large files.
Input files can also be manipulated using \LangmuirPython, as discussed
    in section ~\ref{sec:python}.
In \Langmuir, input files and checkpoint files are the same thing.
Periodically, a running simulation will save a checkpoint file.
You can use this checkpoint file to extend the simulation or change
    its parameters.
The \verb|#| symbol serves as a comment inside the input file.
Any text after the \verb|#| symbol is ignored.
The input file is divided into sections.
Sections always start with a header.
The valid section headers are shown below.

\begin{itemize}
    \item \verb|[Electrons]|
    \item \verb|[Holes]|
    \item \verb|[Defects]|
    \item \verb|[Traps]|
    \item \verb|[TrapPotentials]|
    \item \verb|[FluxState]|
    \item \verb|[RandomState]|
    \item \verb|[Parameters]|
\end{itemize}

The only required section is the \verb|[Parameters]| section,
    and it must be the last section in the input file.
Other sections do not have to be in any particular order.
Example input files are found in section~\ref{ssec:examples}.

\subsection{Site IDs}
    \label{ssec:sites}
    The position of an agent in the grid can be thought of as a 3-tuple
        of integers ($x_{i}$, $y_{i}$, $z_{i}$).
    This 3-tuple can be hashed into a single number called
        the site-id, $s_{i}$.
    The dimensions of the grid are $L_{x}$, $L_{y}$, and $L_{z}$.
    Note that $L_{x}$, $L_{y}$, and $L_{z}$ are the \verb|grid.x|,
        \verb|grid.y|, and \verb|grid.z| parameters discussed in
        section~\ref{ssec:parameters}.
    The following equations hold for site-ids, where all quantities
        are integers, and integer division applies.

    \begin{alignat}{3}
        \label{eqn:xrange}
        & 0 \leq & x_{i} < & L_{x} & x_{i} =& s_{i} \% L_{x} \\
        \label{eqn:yrange}
        & 0 \leq & y_{i} < & L_{y} & y_{i} =& s_{i} / L_{x} - (s_{i}/(L_{x}L_{y})) L_{y} \\
        \label{eqn:zrange}
        & 0 \leq & z_{i} < & L_{z} & z_{i} =& s_{i} / (L_{x}L_{y}) \\
        \label{eqn:srange}
        & 0 \leq & s_{i} < & L_{x} L_{y} L_{z} \quad & s_{i} =& L_{x} (y_{i} + z_{i} L_{y})+ x_{i}
    \end{alignat}

\newpage
\subsection{Agents}
    \label{ssec:agents}
    Electrons, holes, defects, and traps all follow the same format.
    Note that these sections are for providing information on
        electrons, etc.\ already present in the system before the
        simulation starts.
    This is typically the case when extending a run, or placing
        traps at well defined locations.
    You may leave these sections out.
    \Langmuir has the ability to place traps, defects, and carriers
        randomly if desired (see section~\ref{ssec:parameters}).

    It is very important that the parameters
        \verb|electron.percentage|, \verb|hole.percentage|,
        \verb|defect.percentage|, and \verb|trap.percentage|
        are consistent with these sections.
    For example, while the number of electrons in the \verb|Electrons|
        section can be less than the maximum number of electrons
        allowed by \verb|electron.percentage|, it can not exceed
        the max.
    If there is a problem, \langmuir will raise an error.

    The first line is the section header written in square brakets.
    The next line is always the number of elements to be read by
        \langmuir.
    For example, for the electrons, the second line is the number of
        electrons.
    The remaining lines are the site-ids for electrons, holes, defects,
        and traps.
    Site-id's are discussed in section~\ref{ssec:sites}.

    \begin{bashcode*}{gobble=8}
        [Electrons]    # section header
        2              # number of electrons
        100            # site-id of electron 1
        200            # site-id of electron 2
        [Holes]        # section header
        0              # number of holes
        [Defects]      # section header
        0              # number of defects
        [Traps]        # section header
        0              # number of traps
    \end{bashcode*}

\subsection{Trap Potentials}
    \label{ssec:trap}
    The \verb|[TrapPotentials]| section is very similar in structure to
        the \verb|[Trap]| section.
    The only difference is that instead of site-ids, one lists the
        trap potentials in units of \si{eV}.
    If present, the \verb|[TrapPotentials]| section must be the same
        size as the \verb|[Traps]| section.
    Note that if all traps have the same value, then this section
        can be omitted.
    In this case, the value used for trap potential is taken from the
        \verb|trap.potential| parameter (see section~\ref{ssec:parameters}).

    \begin{bashcode*}{gobble=8}
        [TrapPotentials]    # section header
        2                   # number of traps
        0.50                # trap potential of trap 1
        0.50                # trap potential of trap 2
    \end{bashcode*}

\newpage
\subsection{Flux State}
    \label{ssec:flux}
    The \verb|[FluxState]| is a list of 20 integers detailing the
        number of attempts and successes made by a flux agent.
    Examples of flux agents are the sources and drains.
    There are 10 different flux agents in \Langmuir.
    There are 2 source agents and 2 drain agents at $x_{i} = 0$ and
        $x_{i} = L_{x} - 1$, making a total of 8.
    The remaining 2 are an exciton source, and a recombination drain.
    You probably never have to edit this section.

    \begin{bashcode*}{gobble=8}
        # The codes should hopefully be easy to figure out...
        # ESLA = Electron Source Left Attempt
        # XDS  = Recombination Drain Success
        # etc...
        [FluxState]    # section header
        20             # number of flux agents
        0              # ESLA
        0              # ESLS
        0              # ESRA
        0              # ESRS
        0              # HSLA
        0              # HSLS
        0              # HSRA
        0              # HSRS
        0              # XSA
        0              # XSS
        0              # EDLA
        0              # EDLS
        0              # EDRA
        0              # EDRS
        0              # HDLA
        0              # HDLS
        0              # HDRA
        0              # HDRS
        0              # XDA
        0              # XDS
    \end{bashcode*}

\subsection{Random State}
    \label{ssec:random}
    The \verb|[RandomState]| is a very long list of integers that
        describe the exact state of the random number generator.
    Due to limitations of the combination of boost, stdlib, and qt,
        it must be on one line.
    You should never have to edit this section, other than deleting it.

    \begin{bashcode*}{gobble=8}
        [RandomState] # section header
        1371835351 1524755492 3319441753 617340572... # list of numbers
    \end{bashcode*}

\newpage
\subsection{Parameters}
    \label{ssec:parameters}
    The parameters section is a list of \verb|key=value| pairs that
        alter the behavior of the simulation.
    This section will often be the only section in an input file.
    It must be present, and it must be the last section in the text
        file.
    Below is a list of parameters and their descriptions.
    
    \bgroup
\NewDocumentCommand{\titles}{}{%
    keyword & type & default & description \\
    \tabucline[1pt]{-}
}
\NewDocumentCommand{\parameter}{m m m m}{%
    \texttt{#1} & \textbf{#2} & #3 & #4 \\
    \tabucline{-}
}
\vspace{\baselineskip}
\noindent
\tabulinesep=8pt
\extrarowsep=1pt
\begin{tabu} to \textwidth {|[1pt]X[3lm]|X[1rm]|X[1rm]|X[5lm]|[1pt]}
\savetabu{parm}
\tabucline[1pt]{-}
\titles
\parameter{simulation.type}{string}{transistor}{%
    solarcell or transistor - changes the behavior of the sources and drains.
}
\parameter{random.seed}{int}{0}{%
    if 0, then use the current time, else seed the random number generator.
}
\tabucline[1pt]{-}
\end{tabu}

\vspace{\baselineskip}
\noindent
\begin{tabu}{\usetabu{parm}}
\tabucline[1pt]{-}
\titles
\parameter{grid.z}{int}{1}{%
    The height of the device, or number of sites in the z-direction (layers).
}
\parameter{grid.y}{int}{128}{%
    The width of device, or number of sites in the y-direction.
}
\parameter{grid.x}{int}{128}{%
    The length of device, or number of sites in the x-direction
        (source to drain).
}
\parameter{hopping.range}{int}{1}{%
    The number of adjacent sites to consider as neighbors when hopping.
}
\tabucline[1pt]{-}
\end{tabu}

\vspace{\baselineskip}
\noindent
\begin{tabu}{\usetabu{parm}}
\tabucline[1pt]{-}
\titles
\parameter{iterations.real}{int}{1000}{%
    The number of simulation steps, including equilibration.
    It is up to you to remove the equilibration steps from the output.
}
\parameter{iterations.print}{int}{10}{%
    The number of steps between printing output
}
\parameter{current.step}{int}{0}{%
    The starting step of the simulation.  Needed for checkpoint files.
}
\tabucline[1pt]{-}
\end{tabu}

\vspace{\baselineskip}
\noindent
\begin{tabu}{\usetabu{parm}}
\tabucline[1pt]{-}
\titles
\parameter{output.is.on}{bool}{True}{%
    Create output files.
    It is useful to turn off the output when using \LangmuirView.
}
\parameter{output.precision}{int}{15}{%
    The number of digits to print for numbers in various output files.
}
\parameter{output.width}{int}{23}{%
    The width of columns in the output file.
}
\parameter{output.stub}{string}{out}{%
    The naming scheme of output files.
    For example, if stub is ``out'', then the output files are ``out.dat'',
    ``out.chk'', etc.
}
\parameter{output.ids.on.delete}{bool}{False}{%
    Save carrier lifetime and path length to a file when the carrier reaches
    a drain.
    This can make very large files.
}
\parameter{output.ids.on.encounter}{bool}{False}{%
    Save carrier lifetime and path length to a file when the carrier forms
    an exciton.
    This can make very large files.
}
\parameter{output.coulomb}{int}{0}{%
    Output the Coulomb energy of the entire grid every \texttt{iterations.print}
        $\times$ \texttt{output.coulomb} steps.
    If \texttt{output.coulomb} $<$ 0, then save the Coulomb energy when then
        the simulation finishes.
    This requires OpenCL.
    If the grid is too large it may not work if the GPU is too small.
}
\parameter{output.step.chk}{int}{1}{%
    Output checkpoint files every \texttt{iterations.print} $\times$
        \texttt{output.step.chk}.
    When there is a large number of trap sites, writing checkpoint files
        will slow the simulation down.
    Use this parameter to make sure checkpoint files are written far less often
        than the \texttt{iterations.print} value.
}
\parameter{output.chk.trap.potential}{bool}{False}{%
    Suppress the writing of trap potentials to the checkpoint file.
    It is redundant and slow to output trap potentials when they are all
        the same value.
}
\parameter{output.potential}{bool}{False}{%
    Output the potential of the entire grid at the start of the simulation.
    This grid potential does not include the trap potential or the Coulomb
        interactions.
}
\tabucline[1pt]{-}
\end{tabu}

\vspace{\baselineskip}
\noindent
\begin{tabu}{\usetabu{parm}}
\tabucline[1pt]{-}
\titles
\parameter{output.xyz}{int}{0}{%
    Output carrier locations to an xyz file every \texttt{iterations.print}
    $\times$ \texttt{output.xyz}.
    This file will be large.
    This file will not open easily in VMD without the use of a VMD extension
        because the number of particles can change.
    There is a \texttt{vmd.init} file in the \Langmuir source directory to
        help with opening this file.
}
\parameter{output.xyz.e}{bool}{True}{%
    Output the electrons to the xyz file.
}
\parameter{output.xyz.h}{bool}{True}{%
    Output the holes to the xyz file.
}
\parameter{output.xyz.d}{bool}{True}{%
    Output the defects to the xyz file.
}
\parameter{output.xyz.t}{bool}{True}{%
    Output the traps to the xyz file.
    When there are tons of traps, the size of the xyz file can become
        too large to handle.
    You should suppress the output of traps to the xyz file in this case.
}
\parameter{output.xyz.mode}{int}{0}{%
    When 0, the number of particles between frames in the xyz file can vary.
    If 1, the number of particles is kept constant using ``phantom particles''
}
\tabucline[1pt]{-}
\end{tabu}

\vspace{\baselineskip}
\noindent
\begin{tabu}{\usetabu{parm}}
\tabucline[1pt]{-}
\titles
\parameter{image.traps}{bool}{False}{%
    Save a png of the traps at the start.
    Assumes \texttt{grid.z} = 1.
}
\parameter{image.defects}{bool}{False}{%
    Save a png of the defects at the start.
    Assumes \texttt{grid.z} = 1.
}
\parameter{image.carriers}{int}{0}{%
    Save a png of the carriers every \texttt{iterations.print} $\times$
        \texttt{image.carriers}.
    If \texttt{image.carriers} $<$ 0, then save the png when then
        the simulation finishes.
    Assumes \texttt{grid.z} = 1.
}
\tabucline[1pt]{-}
\end{tabu}

\vspace{\baselineskip}
\noindent
\begin{tabu}{\usetabu{parm}}
\tabucline[1pt]{-}
\titles
\parameter{electron.percentage}{float}{0.01}{%
    Sets the maximum number of allowed electrons to be the volume of the grid
        times this percentage.
    Between 0 and 1.
}
\parameter{hole.percentage}{float}{0.0}{%
    Sets the maximum number of allowed holes to be the volume of the grid
        times this percentage.
    Between 0 and 1.
}
\parameter{seed.charges}{float}{0.0}{%
    The fraction of the maximum electrons/holes to place randomly at the
        beginning of the simulation.
    Between 0 and 1.
    This helps with equilibration in transistors.
    Have not tested this in solar cells.
}
\tabucline[1pt]{-}
\end{tabu}

\vspace{\baselineskip}
\noindent
\begin{tabu}{\usetabu{parm}}
\tabucline[1pt]{-}
\titles
\parameter{defect.percentage}{float}{0.0}{%
    Sets the maximum number of defects to be the volume of the grid
        times this percentage.
    Between 0 and 1.
    Defects are placed randomly at the start.
}
\parameter{defects.charge}{int}{0}{%
    The charge of defects.
    If 0, then defects are not included in Coulomb calculations.
}
\tabucline[1pt]{-}
\end{tabu}

\vspace{\baselineskip}
\noindent
\begin{tabu}{\usetabu{parm}}
\tabucline[1pt]{-}
\titles
\parameter{trap.percentage}{float}{0.0}{%
    Sets the maximum number of traps to be the volume of the grid
        times this percentage.
    Between 0 and 1.
    Traps are placed randomly.
}
\parameter{seed.percentage}{float}{1.0}{%
    The fraction of the traps to place as seeds.
    Remaining traps are grown around these seeds.
    Between 0 and 1.
}
\parameter{trap.potential}{float}{0.1}{%
    The trap energy to use for randomly placed traps.
}
\parameter{gaussian.stdev}{float}{0.0}{%
    Standard deviations of random noise to be added to randomly placed traps.
}
\tabucline[1pt]{-}
\end{tabu}

\vspace{\baselineskip}
\noindent
\begin{tabu}{\usetabu{parm}}
\tabucline[1pt]{-}
\titles
\parameter{voltage.right}{float}{0.0}{%
    The voltage of the drain electrode.
}
\parameter{voltage.left}{float}{0.0}{%
    The voltage of the source electrode.
    Keep this zero and alter \texttt{voltage.right}.
}
\parameter{exciton.binding}{float}{0.0}{%
    The energy of interaction when a hole and electron are on the same site.
}
\parameter{slope.z}{float}{0.0}{%
    The voltage change along the z direction due to a gate electrode.
}
\parameter{coulomb.carriers}{bool}{False}{%
    Turn on Coulomb interactions.
}
\parameter{coulomb.gaussian.sigma}{float}{0.0}{%
    The standard deviation of smeared out Gaussian charges.
    If 0, then point charges are used.
    Assumes \texttt{grid.z} $>$ 1.
}
\parameter{temperature.kelvin}{float}{300.0}{%
    The temperature used in the Boltzmann factor.
}
\tabucline[1pt]{-}
\end{tabu}

\vspace{\baselineskip}
\noindent
\begin{tabu}{\usetabu{parm}}
\tabucline[1pt]{-}
\titles
\parameter{source.rate}{float}{0.9}{%
    Default probability to inject charges.
    Between 0 and 1.
}
\parameter{e.source.l.rate}{float}{-1.0}{%
    Injection rate of electrons from the left.
    Overrides \texttt{source.rate}.
    Ignored if $<$ 0.
}
\parameter{e.source.r.rate}{float}{-1.0}{%
    Injection rate of electrons from the right.
    Overrides \texttt{source.rate}.
    Ignored if $<$ 0.
}
\parameter{h.source.l.rate}{float}{-1.0}{%
    Injection rate of holes from the left.
    Overrides \texttt{source.rate}.
    Ignored if $<$ 0.
}
\parameter{h.source.r.rate}{float}{-1.0}{%
    Injection rate of holes from the right.
    Overrides \texttt{source.rate}.
    Ignored if $<$ 0.
}
\parameter{generation.rate}{float}{0.001}{%
    Injection rate of excitons.
    Overrides \texttt{source.rate}.
    Ignored if $<$ 0.
}
\parameter{balance.charges}{bool}{False}{%
    Try to keep the number of electrons and holes equal.
    Not physical.
}
\parameter{source.metropolis}{bool}{False}{%
    Override source injection probability with a metropolis criterion
        involving site energy.
}
\parameter{source.coulomb}{bool}{False}{%
    Include coulomb interactions with image charges in the metropolis
        criterion.
}
\parameter{source.scale.area}{float}{65536.0}{%
    Scale the generation rate by dividing by this value and multiplying by
        the xy-area of the system.
}
\tabucline[1pt]{-}
\end{tabu}

\vspace{\baselineskip}
\noindent
\begin{tabu}{\usetabu{parm}}
\tabucline[1pt]{-}
\titles
\parameter{drain.rate}{float}{0.9}{%
    Default probability to accept charges.
    Between 0 and 1.
}
\parameter{e.drain.l.rate}{float}{-1.0}{%
    Acceptance rate of electrons on the left.
    Overrides \texttt{drain.rate}.
    Ignored if $<$ 0.
}
\parameter{e.drain.r.rate}{float}{-1.0}{%
    Acceptance rate of electrons on the right.
    Overrides \texttt{drain.rate}.
    Ignored if $<$ 0.
}
\parameter{h.drain.l.rate}{float}{-1.0}{%
    Acceptance rate of holes on the left.
    Overrides \texttt{drain.rate}.
    Ignored if $<$ 0.
}
\parameter{h.drain.r.rate}{float}{-1.0}{%
    Acceptance rate of holes on the right.
    Overrides \texttt{drain.rate}.
    Ignored if $<$ 0.
}
\parameter{recombination.rate}{float}{0.0}{%
    Probability to recombine excitons.
    Note - it is not really a rate like the others because
    the number of excitons in the system is hard to predict.
}
\parameter{recombination.range}{int}{0}{%
    Number of adjacent sites to consider during recombination.
}
\tabucline[1pt]{-}
\end{tabu}

\vspace{\baselineskip}
\noindent
\begin{tabu}{\usetabu{parm}}
\tabucline[1pt]{-}
\titles
\parameter{use.opencl}{bool}{False}{%
    Use OpenCL for Coulomb calculations.
}
\parameter{work.x}{int}{4}{%
    The number of x-threads in a 3D work group.
    Only used for \texttt{output.coulomb}.
    The total size of a work group is
        $W = \mathtt{work.x} \times \mathtt{work.y} \times \mathtt{work.z}$.
    The total size of the grid is
        $G = \mathtt{grid.x} \times \mathtt{grid.y} \times \mathtt{grid.z}$.
    The total number of threads used by the 3D kernel is $T = G\times W$.
    The 3D kernel will fail if you exceed the limitations of the GPU.
    This could be fixed by dividing the grid into sections and using
        multiple GPU's or multiple calls to one GPU.
    The max $W$ allowed on GTX460 is $1024$.
    The max $T$ allowed on GTX460 is $1024 \times 1024 \times 64$.
    Therefore, the maximum number of grid sites that could be handled is
        $65536 = 256^{2}$.
}
\parameter{work.y}{int}{4}{%
    The number of y-threads in a 3D work group.
    See \texttt{work.x} for more info.
}
\parameter{work.z}{int}{4}{%
    The number of z-threads in a 3D work group.
    See \texttt{work.x} for more info.
}
\parameter{work.size}{int}{256}{%
    The number of threads $W$ in a 1D work group.
    The total number of threads used during a coulomb calculation is
        $T = N\times W$, where $N$ is the total number of charges.
    It is unlikely you will exceed the total number of allowed threads
        on the GPU.
    For magical reasons, this parameter seems to be optimal at $256$.
    The maximum value of $W$ on a GTX460 is $1024$.
    The maximum number of threads on a GTX460 is $1024\times1024\times64$.
    Therefore, the maximum number of charges allowed when $W = 256$ is
        $N = 262144$.
}
\parameter{opencl.threshold}{int}{256}{%
    The number of charges that must be present before turning on OpenCL.
    OpenCL will be slower than the CPU for small numbers of charges.
}
\parameter{opencl.device.id}{int}{0}{%
    The id of the GPU to use when more than one is present.  Currently
    this parameter is ignored.  The GPU chosen defaults to 0, unless
    a PBS\_GPUFILE is found, in which case the GPU used is chosen by PBS.
    The id of the GPU chosen is saved to this variable.
}
\parameter{max.threads}{int}{-1}{%
    The max number of CPU threads allowed.
    If $<$ 0, then use the number of threads recommended by QtConcurrent.
    However, if $<0$ and a PBS\_NODEFILE is found, then the number of threads
        is chosen by PBS.
    \LangmuirView currently ignores this parameter and uses the number
        of threads recommended by QtConcurrent.
}
\tabucline[1pt]{-}
\end{tabu}

\egroup


\newpage
\subsection{Examples}
    \label{ssec:examples}
    This section contains sample input files.

    \subsubsection{Transistor}
    \begin{bashcode*}{gobble=8}
        [Parameters]
        simulation.type         = transistor

        grid.x                  = 1024
        grid.y                  = 256
        grid.z                  = 1

        iterations.real         = 500000
        iterations.print        = 1000

        electron.percentage     = 0.10
        seed.charges            = 1.00      # speed up equilibration

        voltage.right           = 5.00
        voltage.left            = 0.00

        coulomb.carriers        = true
        use.opencl              = true
    \end{bashcode*}

    \subsubsection{Solar Cell}
    \begin{bashcode*}{gobble=8}
        [Parameters]
        simulation.type         = solarcell

        grid.x                  = 256
        grid.y                  = 256
        grid.z                  = 1

        iterations.real         = 20000000  # much longer than transistor
        iterations.print        = 1000

        electron.percentage     = 0.10
        hole.percentage         = 0.10

        trap.percentage         = 0.50
        trap.potential          = 0.50
        seed.percentage         = 0.10

        voltage.right           = 9.00
        voltage.left            = 0.00

        source.rate             = 1e-3
        recombination.rate      = 1e-4

        coulomb.carriers        = true
        use.opencl              = true
    \end{bashcode*}

    \newpage
    \subsubsection{Scan}
    Set a variable equal to a list of values.  This works for any variable.
    See section~\ref{sec:python}.
    \begin{bashcode*}{gobble=8}
        [Parameters]
        ...
        voltage.right = [-2.0, -1.5, -1.0, -0.8, -0.6, -0.4, -0.2, 0.0]
        ...
    \end{bashcode*}   
    To generate input files.
    \begin{bashcode*}{gobble=8}
        adam@work: python scan.py --real 500000 --print 1000 --mode gen
    \end{bashcode*}
    To run \Langmuir in real time.
    \begin{bashcode*}{gobble=8}
        adam@work: python scan.py --real 500000 --print 1000 --mode scan
    \end{bashcode*}
    
    \subsubsection{Traps}
    \begin{bashcode*}{gobble=8}
        [Parameters]
        ...
        trap.percentage         = 0.50 # 50-percent traps
        seed.percentage         = 0.10 # 10-percent seeds
        trap.potential          = 0.05 #
        ...
    \end{bashcode*} 

    \subsubsection{Defects}
    \begin{bashcode*}{gobble=8}
        [Parameters]
        ...
        defect.percentage       = 0.50 # 50-percent defects
        defect.charge           = 0    # neutral defects
        ...
    \end{bashcode*}

    \subsubsection{Coulomb}
    \begin{bashcode*}{gobble=8}
        [Parameters]
        ...
        coulomb.carriers        = true # turn on coulomb interations
        use.opencl              = true # use GPU
        output.coulomb          = 10   # output energy every 10 iterations.print
        ...
    \end{bashcode*}
    To calculate coulomb energy of a checkpoint file you can use the
        python script coulomb.py.
    If the system is too big for the GPU you have to use coulomb.py.
    Or you can run \langmuir again.
    \begin{bashcode*}{gobble=8}
        [Parameters]
        ...
        iterations.real         =  0 # do not simulate anything
        output.coulomb          = -1 # output energy at the end
        ...
    \end{bashcode*}    
    