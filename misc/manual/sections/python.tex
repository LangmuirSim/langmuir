\newpage
\section{Langmuir Python}
\label{sec:python}
There is a python project called \LangmuirPython, that aids in the
    construction of input files, the running
    of simulations, and the analysis of output files.
The use of \LangmuirPython in not required.
There is documentation in the code.
A pdf and/or html webpage of the documentation can be generated with Sphinx.
Simply navigate to the \LangmuirPython doc directory, and use the Makefile.
Sphinx must be installed.

\subsection{Installation}
    \LangmuirPython uses a number of python modules that may not be
        included in standard python distributions.
    You should make an effort to install these python modules, for
        example, using pip.
    If modules are missing, an effort has been made to disable certain
        features of the \langmuir module.
    \begin{itemize}
        \item numpy
        \item scipy
        \item matplotlib
        \item pandas
        \item quantities
        \item vtk (optional)
    \end{itemize}

    To install \LangmuirPython, you must do one of the following.
    \begin{enumerate}
        \item Put the path to LangmuirPython on your \$PYTHONPATH.
            \begin{bashcode*}{gobble=16}
                # set $PYTHONPATH inside batch scripts or bashrc
                export PYTHONPATH=$PYTHONPATH:/path/to/LangmuirPython
            \end{bashcode*}
        \item Use the setup.py script in the LangmuirPython directory.
            \begin{bashcode*}{gobble=16}
                adam@work: cd /path/to/LangmuirPython

                # to install locally
                adam@work: python setup.py install --user

                # to install globally (need root password)
                adam@work: python setup.py install
            \end{bashcode*}
    \end{enumerate}

\subsection{Usage}
    To use the \langmuir module in python you must import it.
    \begin{pythoncode*}{gobble=8}
        >>> import langmuir as lm
    \end{pythoncode*}

    The checkpoint submodule is very useful for manipulating inputs
        file from python.
    \begin{pythoncode*}{gobble=8}
        >>> import langmuir as lm
        >>> chk = lm.checkpoint.load('out.chk')
        >>> chk.electrons = [] # delete electrons
        >>> chk['iterations.real'] += 100000
        >>> chk.save('sim.inp')
    \end{pythoncode*}

    The usage of most scripts can be determined by using the -h or
        --help command line flag.
    \begin{bashcode*}{gobble=8}
        adam@work: python /path/to/LangmuirPython/utils/scan.py --help
    \end{bashcode*}

    Some scripts should be pointed out.
    \begin{itemize}
        \item \verb|/utils/scan.py|
            \begin{bashcode*}{gobble=16}
                # --mode scan actually runs langmuir
                # the output of one simulation serves as the input of the next
                # sim.inp has a working variable set to a list of values
                # example: voltage.right = [1, 2, 3, 4, 5]
                adam@work: python scan.py --real 500000 --mode scan sim.inp

                # --mode gen generates simulations input
                adam@work: python scan.py --real 500000 --mode gen sim.inp
            \end{bashcode*}
        \item \verb|/analyze/combine.py|
            \begin{bashcode*}{gobble=16}
                # combines the output of "parts"
                adam@work: python combine.py -r
            \end{bashcode*}
        \item \verb|/analyze/gather.py|
            \begin{bashcode*}{gobble=16}
                # gathers output from simulations
                # use this inside a "run" directory
                adam@work: python gather.py -r --equil 10000
            \end{bashcode*}
    \end{itemize}

    A lot of the scripts rely on you structuring your simulations.
    \begin{bashcode*}{gobble=8}
        system/
            run.0/
                voltage.right_+0.0/
                    part.0/
                        out.dat
                        out.chk
                    part.1/
                        ...
                    ...
                voltage.right_+0.1/
                ...
            run.1/
            run.2/
            ...
    \end{bashcode*}
