\newpage
\section{Usage}
\label{sec:usage}

\subsection{Langmuir}
    \label{ssec:langmuir}
    \Langmuir is generally run inside a terminal.
    \begin{bashcode*}{gobble=8}
        adam@work: langmuir input.inp
    \end{bashcode*}
    Note that the \langmuir command must be on your path.
    There is only one argument, the full path to the input file.
    However, note that \Langmuir will write output files to the directory it is
        called from.
    The format of the input file is discussed in section~\ref{sec:input}.
    Often you will run \Langmuir on a cluster.
    Details on how to run \Langmuir on a cluster, as well as sample batch
        scripts are included in section~\ref{sec:batch}.

\subsection{LangmuirView}
    \label{ssec:langmuirview}
    \LangmuirView is used to watch simulations graphically in real
        time with a GUI.
    \LangmuirView can be run from a terminal.
    \begin{bashcode*}{gobble=8}
        adam@work: langmuirView input.inp
    \end{bashcode*}    
    Note that the \langmuirView command must be on your path.
    Unlike the \langmuir command, the input file argument is optional.
    If no input file is given, then \LangmuirView will open a file dialogue
        for you to choose the location of an input file.
    An example of \LangmuirView is shown in Figure~\ref{fig:langmuirview}.
    \begin{figure}[H]
        \centering
        \includegraphics[scale=0.3]{/langmuirView_0.png}
        \caption{\label{fig:langmuirview}
            LangmuirView example.
        }
    \end{figure}
    \LangmuirView is a tool to be used to aid in understanding
        and communication to others, for example, to make movies or
        screen shots.
    To perform actual simulations you should use \Langmuir from the
        command line, as discussed in section ~\ref{ssec:langmuir}.
    This is because \LangmuirView is limited by the size of the
        simulation and the production of output data.
    Do not attempt to use \LangmuirView with large systems that
        produce a large amount of output data.
