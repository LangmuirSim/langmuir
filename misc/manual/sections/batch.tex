\newpage
\section{Batch Files}
\label{sec:batch}
Details on how to run \Langmuir on a cluster are discussed in this section.

\subsection{Cluster Commands}
\label{ssec:cluster}

\begin{bashcode*}{gobble=4}
    # show the jobs
    qstat
\end{bashcode*}

\begin{bashcode*}{gobble=4}
    # submit a batch script
    qsub -N jobname run.batch
\end{bashcode*}

\subsection{Hutchison}
\label{ssec:hutch}
On \url{hutchison.chem.pitt.edu}, one can use the submitLangmuir command.

\begin{bashcode*}{gobble=4}
    submitLangmuir jobname job.inp
\end{bashcode*}

\noindent
Or you can use a batch script such as the one below.
\begin{bashcode*}{gobble=4}
    #!/bin/sh                                                                                                                                                                                                    
    #$ -S /bin/bash
    #$ -pe mpi 4
    #$ -R y
    #$ -cwd
    
    INPUT=$1
    WORK=$2
    BIN=/usr/local/bin
    EXE=langmuir
    SCRATCH=/scratch/${USER}/${JOB_ID}
    
    if [ -d /scratch/${USER} ]; then 
    touch /scratch/${USER}
    else
    mkdir /scratch/${USER}
    fi
    
    mkdir ${SCRATCH}
    
    if [ -f ${INPUT} ]; then 
    cp ${INPUT} ${SCRATCH}
    else
    exit 
    fi
    
    cd ${SCRATCH}
    ${BIN}/${EXE} ${INPUT}
    
    gzip ${SCRATCH}/* -rv
    cp -r ${SCRATCH}/* ${WORK}
\end{bashcode*}

\newpage
\subsection{Frank}
On \url{frank.sam.pitt.edu}, use the following batch script.
Make sure to set the path to langmuir in the \verb|BIN| variable.

\begin{bashcode*}{gobble=4}
    #!/bin/bash
    #PBS -q gpu
    #PBS -l nodes=1:ppn=3:gpus=1
    #PBS -l mem=4gb
    #PBS -l walltime=72:00:00
    BIN=/home/ghutchison/agg7/bin
    EXE=langmuir
    NME=sim.inp
    module load cuda/4.2
    module load boost
    module load qt
    cd $PBS_O_WORKDIR
    $BIN/$EXE $NME
\end{bashcode*}

\subsection{Scan}
Sometimes, instead of running a single \langmuir simulation, one may want
    to scan over many values of a variable.
In the \LangmuirPython module (see section~\ref{sec:python}), there is a script
    called scan.py.
This script (scan.py) lets python guide the scanning of a working variable.
For example, one can scan \verb|voltage.right| to create an IV curve.
Instead of setting a variable in the input file to a single value, set it to
    a list of values.
\begin{bashcode*}{gobble=4}
    voltage.right = [-1.0, -2.0, -3.0]
\end{bashcode*}
Actually, scan.py can parse any valid python statement, including numpy.
\begin{bashcode*}{gobble=4}
    voltage.right = [float(i) for i in range(0, 100, 10)]
\end{bashcode*}
\begin{bashcode*}{gobble=4}
    voltage.right = np.linspace(0, 100, 10)
\end{bashcode*}
Multiple working variables are not supported.
You must use an altered batch script to run scan.py.
Make sure the script correctly sets the \verb|$PYTHONPATH| to point to
    \LangmuirPython.
Below is an example batch script for \verb|frank.sam.pitt.edu|.
\begin{bashcode*}{gobble=4}
    #!/bin/bash                  
    #PBS -q gpu                                                                            
    #PBS -l nodes=1:ppn=4:gpus=1                                                           
    #PBS -l mem=4gb                                                                        
    #PBS -l walltime=36:00:00                                                              
                                                                                        
    BIN=/home/ghutchison/agg7/bin                                                          
    EXE=scan.py                                                                            
    NME=sim.inp                                                                            
                                                                                        
    module load cuda/4.2                                                                   
    module load python                                                                     
    module load boost                                                                      
    module load qt                                                                         
                                                                                        
    export PYTHONPATH=$PYTHONPATH:/home/ghutchison/agg7/LangmuirPython                     
                                                                                        
    cd $PBS_O_WORKDIR                                                                      
    python $BIN/$EXE --real 7500000 --print 1000 --fmt '%+.1f' --mode scan
\end{bashcode*}

