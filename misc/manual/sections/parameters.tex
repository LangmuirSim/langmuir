\bgroup
\NewDocumentCommand{\titles}{}{%
    keyword & type & default & description \\
    \tabucline[1pt]{-}
}
\NewDocumentCommand{\parameter}{m m m m}{%
    \texttt{#1} & \textbf{#2} & #3 & #4 \\
    \tabucline{-}
}
\vspace{\baselineskip}
\noindent
\tabulinesep=8pt
\extrarowsep=1pt
\begin{tabu} to \textwidth {|[1pt]X[3lm]|X[1rm]|X[1rm]|X[5lm]|[1pt]}
\savetabu{parm}
\tabucline[1pt]{-}
\titles
\parameter{simulation.type}{string}{transistor}{%
    solarcell or transistor - changes the behavior of the sources and drains.
}
\parameter{random.seed}{int}{0}{%
    if 0, then use the current time, else seed the random number generator.
}
\tabucline[1pt]{-}
\end{tabu}

\vspace{\baselineskip}
\noindent
\begin{tabu}{\usetabu{parm}}
\tabucline[1pt]{-}
\titles
\parameter{grid.z}{int}{1}{%
    The height of the device, or number of sites in the z-direction (layers).
}
\parameter{grid.y}{int}{128}{%
    The width of device, or number of sites in the y-direction.
}
\parameter{grid.x}{int}{128}{%
    The length of device, or number of sites in the x-direction
        (source to drain).
}
\parameter{hopping.range}{int}{1}{%
    The number of adjacent sites to consider as neighbors when hopping.
}
\tabucline[1pt]{-}
\end{tabu}

\vspace{\baselineskip}
\noindent
\begin{tabu}{\usetabu{parm}}
\tabucline[1pt]{-}
\titles
\parameter{iterations.real}{int}{1000}{%
    The number of simulation steps, including equilibration.
    It is up to you to remove the equilibration steps from the output.
}
\parameter{iterations.print}{int}{10}{%
    The number of steps between printing output
}
\parameter{current.step}{int}{0}{%
    The starting step of the simulation.  Needed for checkpoint files.
}
\tabucline[1pt]{-}
\end{tabu}

\vspace{\baselineskip}
\noindent
\begin{tabu}{\usetabu{parm}}
\tabucline[1pt]{-}
\titles
\parameter{output.is.on}{bool}{True}{%
    Create output files.
    It is useful to turn off the output when using \LangmuirView.
}
\parameter{output.precision}{int}{15}{%
    The number of digits to print for numbers in various output files.
}
\parameter{output.width}{int}{23}{%
    The width of columns in the output file.
}
\parameter{output.stub}{string}{out}{%
    The naming scheme of output files.
    For example, if stub is ``out'', then the output files are ``out.dat'',
    ``out.chk'', etc.
}
\parameter{output.ids.on.delete}{bool}{False}{%
    Save carrier lifetime and path length to a file when the carrier reaches
    a drain.
    This can make very large files.
}
\parameter{output.ids.on.encounter}{bool}{False}{%
    Save carrier lifetime and path length to a file when the carrier forms
    an exciton.
    This can make very large files.
}
\parameter{output.coulomb}{int}{0}{%
    Output the Coulomb energy of the entire grid every \texttt{iterations.print}
        $\times$ \texttt{output.coulomb} steps.
    If \texttt{output.coulomb} $<$ 0, then save the Coulomb energy when then
        the simulation finishes.
    This requires OpenCL.
    If the grid is too large it may not work if the GPU is too small.
}
\parameter{output.step.chk}{int}{1}{%
    Output checkpoint files every \texttt{iterations.print} $\times$
        \texttt{output.step.chk}.
    When there is a large number of trap sites, writing checkpoint files
        will slow the simulation down.
    Use this parameter to make sure checkpoint files are written far less often
        than the \texttt{iterations.print} value.
}
\parameter{output.chk.trap.potential}{bool}{False}{%
    Suppress the writing of trap potentials to the checkpoint file.
    It is redundant and slow to output trap potentials when they are all
        the same value.
}
\parameter{output.potential}{bool}{False}{%
    Output the potential of the entire grid at the start of the simulation.
    This grid potential does not include the trap potential or the Coulomb
        interactions.
}
\tabucline[1pt]{-}
\end{tabu}

\vspace{\baselineskip}
\noindent
\begin{tabu}{\usetabu{parm}}
\tabucline[1pt]{-}
\titles
\parameter{output.xyz}{int}{0}{%
    Output carrier locations to an xyz file every \texttt{iterations.print}
    $\times$ \texttt{output.xyz}.
    This file will be large.
    This file will not open easily in VMD without the use of a VMD extension
        because the number of particles can change.
    There is a \texttt{vmd.init} file in the \Langmuir source directory to
        help with opening this file.
}
\parameter{output.xyz.e}{bool}{True}{%
    Output the electrons to the xyz file.
}
\parameter{output.xyz.h}{bool}{True}{%
    Output the holes to the xyz file.
}
\parameter{output.xyz.d}{bool}{True}{%
    Output the defects to the xyz file.
}
\parameter{output.xyz.t}{bool}{True}{%
    Output the traps to the xyz file.
    When there are tons of traps, the size of the xyz file can become
        too large to handle.
    You should suppress the output of traps to the xyz file in this case.
}
\parameter{output.xyz.mode}{int}{0}{%
    When 0, the number of particles between frames in the xyz file can vary.
    If 1, the number of particles is kept constant using ``phantom particles''
}
\tabucline[1pt]{-}
\end{tabu}

\vspace{\baselineskip}
\noindent
\begin{tabu}{\usetabu{parm}}
\tabucline[1pt]{-}
\titles
\parameter{image.traps}{bool}{False}{%
    Save a png of the traps at the start.
    Assumes \texttt{grid.z} = 1.
}
\parameter{image.defects}{bool}{False}{%
    Save a png of the defects at the start.
    Assumes \texttt{grid.z} = 1.
}
\parameter{image.carriers}{int}{0}{%
    Save a png of the carriers every \texttt{iterations.print} $\times$
        \texttt{image.carriers}.
    If \texttt{image.carriers} $<$ 0, then save the png when then
        the simulation finishes.
    Assumes \texttt{grid.z} = 1.
}
\tabucline[1pt]{-}
\end{tabu}

\vspace{\baselineskip}
\noindent
\begin{tabu}{\usetabu{parm}}
\tabucline[1pt]{-}
\titles
\parameter{electron.percentage}{float}{0.01}{%
    Sets the maximum number of allowed electrons to be the volume of the grid
        times this percentage.
    Between 0 and 1.
}
\parameter{hole.percentage}{float}{0.0}{%
    Sets the maximum number of allowed holes to be the volume of the grid
        times this percentage.
    Between 0 and 1.
}
\parameter{seed.charges}{float}{0.0}{%
    The fraction of the maximum electrons/holes to place randomly at the
        beginning of the simulation.
    Between 0 and 1.
    This helps with equilibration in transistors.
    Have not tested this in solar cells.
}
\tabucline[1pt]{-}
\end{tabu}

\vspace{\baselineskip}
\noindent
\begin{tabu}{\usetabu{parm}}
\tabucline[1pt]{-}
\titles
\parameter{defect.percentage}{float}{0.0}{%
    Sets the maximum number of defects to be the volume of the grid
        times this percentage.
    Between 0 and 1.
    Defects are placed randomly at the start.
}
\parameter{defects.charge}{int}{0}{%
    The charge of defects.
    If 0, then defects are not included in Coulomb calculations.
}
\tabucline[1pt]{-}
\end{tabu}

\vspace{\baselineskip}
\noindent
\begin{tabu}{\usetabu{parm}}
\tabucline[1pt]{-}
\titles
\parameter{trap.percentage}{float}{0.0}{%
    Sets the maximum number of traps to be the volume of the grid
        times this percentage.
    Between 0 and 1.
    Traps are placed randomly.
}
\parameter{seed.percentage}{float}{1.0}{%
    The fraction of the traps to place as seeds.
    Remaining traps are grown around these seeds.
    Between 0 and 1.
}
\parameter{trap.potential}{float}{0.1}{%
    The trap energy to use for randomly placed traps.
}
\parameter{gaussian.stdev}{float}{0.0}{%
    Standard deviations of random noise to be added to randomly placed traps.
}
\tabucline[1pt]{-}
\end{tabu}

\vspace{\baselineskip}
\noindent
\begin{tabu}{\usetabu{parm}}
\tabucline[1pt]{-}
\titles
\parameter{voltage.right}{float}{0.0}{%
    The voltage of the drain electrode.
}
\parameter{voltage.left}{float}{0.0}{%
    The voltage of the source electrode.
    Keep this zero and alter \texttt{voltage.right}.
}
\parameter{exciton.binding}{float}{0.0}{%
    The energy of interaction when a hole and electron are on the same site.
}
\parameter{slope.z}{float}{0.0}{%
    The voltage change along the z direction due to a gate electrode.
}
\parameter{coulomb.carriers}{bool}{False}{%
    Turn on Coulomb interactions.
}
\parameter{coulomb.gaussian.sigma}{float}{0.0}{%
    The standard deviation of smeared out Gaussian charges.
    If 0, then point charges are used.
    Assumes \texttt{grid.z} $>$ 1.
}
\parameter{temperature.kelvin}{float}{300.0}{%
    The temperature used in the Boltzmann factor.
}
\tabucline[1pt]{-}
\end{tabu}

\vspace{\baselineskip}
\noindent
\begin{tabu}{\usetabu{parm}}
\tabucline[1pt]{-}
\titles
\parameter{source.rate}{float}{0.9}{%
    Default probability to inject charges.
    Between 0 and 1.
}
\parameter{e.source.l.rate}{float}{-1.0}{%
    Injection rate of electrons from the left.
    Overrides \texttt{source.rate}.
    Ignored if $<$ 0.
}
\parameter{e.source.r.rate}{float}{-1.0}{%
    Injection rate of electrons from the right.
    Overrides \texttt{source.rate}.
    Ignored if $<$ 0.
}
\parameter{h.source.l.rate}{float}{-1.0}{%
    Injection rate of holes from the left.
    Overrides \texttt{source.rate}.
    Ignored if $<$ 0.
}
\parameter{h.source.r.rate}{float}{-1.0}{%
    Injection rate of holes from the right.
    Overrides \texttt{source.rate}.
    Ignored if $<$ 0.
}
\parameter{generation.rate}{float}{0.001}{%
    Injection rate of excitons.
    Overrides \texttt{source.rate}.
    Ignored if $<$ 0.
}
\parameter{balance.charges}{bool}{False}{%
    Try to keep the number of electrons and holes equal.
    Not physical.
}
\parameter{source.metropolis}{bool}{False}{%
    Override source injection probability with a metropolis criterion
        involving site energy.
}
\parameter{source.coulomb}{bool}{False}{%
    Include coulomb interactions with image charges in the metropolis
        criterion.
}
\parameter{source.scale.area}{float}{65536.0}{%
    Scale the generation rate by dividing by this value and multiplying by
        the xy-area of the system.
}
\tabucline[1pt]{-}
\end{tabu}

\vspace{\baselineskip}
\noindent
\begin{tabu}{\usetabu{parm}}
\tabucline[1pt]{-}
\titles
\parameter{drain.rate}{float}{0.9}{%
    Default probability to accept charges.
    Between 0 and 1.
}
\parameter{e.drain.l.rate}{float}{-1.0}{%
    Acceptance rate of electrons on the left.
    Overrides \texttt{drain.rate}.
    Ignored if $<$ 0.
}
\parameter{e.drain.r.rate}{float}{-1.0}{%
    Acceptance rate of electrons on the right.
    Overrides \texttt{drain.rate}.
    Ignored if $<$ 0.
}
\parameter{h.drain.l.rate}{float}{-1.0}{%
    Acceptance rate of holes on the left.
    Overrides \texttt{drain.rate}.
    Ignored if $<$ 0.
}
\parameter{h.drain.r.rate}{float}{-1.0}{%
    Acceptance rate of holes on the right.
    Overrides \texttt{drain.rate}.
    Ignored if $<$ 0.
}
\parameter{recombination.rate}{float}{0.0}{%
    Probability to recombine excitons.
    Note - it is not really a rate like the others because
    the number of excitons in the system is hard to predict.
}
\parameter{recombination.range}{int}{0}{%
    Number of adjacent sites to consider during recombination.
}
\tabucline[1pt]{-}
\end{tabu}

\vspace{\baselineskip}
\noindent
\begin{tabu}{\usetabu{parm}}
\tabucline[1pt]{-}
\titles
\parameter{use.opencl}{bool}{False}{%
    Use OpenCL for Coulomb calculations.
}
\parameter{work.x}{int}{4}{%
    The number of x-threads in a 3D work group.
    Only used for \texttt{output.coulomb}.
    The total size of a work group is
        $W = \mathtt{work.x} \times \mathtt{work.y} \times \mathtt{work.z}$.
    The total size of the grid is
        $G = \mathtt{grid.x} \times \mathtt{grid.y} \times \mathtt{grid.z}$.
    The total number of threads used by the 3D kernel is $T = G\times W$.
    The 3D kernel will fail if you exceed the limitations of the GPU.
    This could be fixed by dividing the grid into sections and using
        multiple GPU's or multiple calls to one GPU.
    The max $W$ allowed on GTX460 is $1024$.
    The max $T$ allowed on GTX460 is $1024 \times 1024 \times 64$.
    Therefore, the maximum number of grid sites that could be handled is
        $65536 = 256^{2}$.
}
\parameter{work.y}{int}{4}{%
    The number of y-threads in a 3D work group.
    See \texttt{work.x} for more info.
}
\parameter{work.z}{int}{4}{%
    The number of z-threads in a 3D work group.
    See \texttt{work.x} for more info.
}
\parameter{work.size}{int}{256}{%
    The number of threads $W$ in a 1D work group.
    The total number of threads used during a coulomb calculation is
        $T = N\times W$, where $N$ is the total number of charges.
    It is unlikely you will exceed the total number of allowed threads
        on the GPU.
    For magical reasons, this parameter seems to be optimal at $256$.
    The maximum value of $W$ on a GTX460 is $1024$.
    The maximum number of threads on a GTX460 is $1024\times1024\times64$.
    Therefore, the maximum number of charges allowed when $W = 256$ is
        $N = 262144$.
}
\parameter{opencl.threshold}{int}{256}{%
    The number of charges that must be present before turning on OpenCL.
    OpenCL will be slower than the CPU for small numbers of charges.
}
\parameter{opencl.device.id}{int}{0}{%
    The id of the GPU to use when more than one is present.  Currently
    this parameter is ignored.  The GPU chosen defaults to 0, unless
    a PBS\_GPUFILE is found, in which case the GPU used is chosen by PBS.
    The id of the GPU chosen is saved to this variable.
}
\parameter{max.threads}{int}{-1}{%
    The max number of CPU threads allowed.
    If $<$ 0, then use the number of threads recommended by QtConcurrent.
    However, if $<0$ and a PBS\_NODEFILE is found, then the number of threads
        is chosen by PBS.
    \LangmuirView currently ignores this parameter and uses the number
        of threads recommended by QtConcurrent.
}
\tabucline[1pt]{-}
\end{tabu}

\egroup
